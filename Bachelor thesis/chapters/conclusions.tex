\chapter*{Conclusions} 
\addcontentsline{toc}{chapter}{Conclusions}

There are several directions for further research on this subject. The bad character heuristic has various versions and can be optimized, giving a different way of solving the problem and the need of verifying if the new methods are correct and complete. The good suffix heuristic can also be used, independently of the bad character heuristic. I started to implement and verify the good suffix heuristic, but I didn't end the proof of it yet.\\
\indent F* was very helpful for my thesis. The lemmas had a crucial role in making the proofs and the refinements helped me in setting certain restrictions and criteria on data types. Another useful element was the \texttt{assert} instruction, which I used to see if the language recognizes a proposition as being valid. One thing that F* did not have but would have been useful is a function that returns the index (or the indices) where an item is stored in a list and the proofs associated with this function. \\
\indent It was interesting to learn a proof-oriented programming language because it was a new concept, different compared to what we did in general in the \(3\) years of faculty. It helped me in improving my critical and analytical thinking and seeing the problems in a different point of view. The proof of the main algorithm was the most interesting part, but at the same time the most complicated part. The proofs were more complex and had more statements and previous lemmas.\\
\indent As statistics, the practical part of the thesis has \(1228\) lines of code, \(114\) lemmas (including \(6\) \texttt{move\_requires} lemmas and \(34\) \texttt{forall\_intro} lemmas), \(85\) \texttt{assert} instructions, \(0\) \texttt{assume} instructions, \(0\) \texttt{admit} instructions and
an execution time of \(69550\) ms (if all of the code is in the same file). \\
\indent Using F*, I proved that my implementation of the Boyer-Moore-Horspool is correct, meaning that, for any \(2\) strings \texttt{text} and \texttt{pattern} with characters in the same alphabet, it computes the first position in the \texttt{text} where we can find the \texttt{pattern}, or \(-1\), if the \texttt{pattern} is not in the \texttt{text}.